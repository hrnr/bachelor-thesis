\chapter{Initial pose estimation problem}
\label{chap:map-merging-intro}

Key problem of map-merging is getting transformation between reference frames of robots. When the transformation is known, merging maps produced by robots is simple. In such a case we can compute transformation to chosen global reference frame and overlay maps in global reference frame. Due to errors in mapping or while mapping in dynamic environment we could get different values across maps for specific global frame coordinates, but this can be solved by taking arithmetic mean, median or extremal values for such coordinates.

Transformation between grids can be acquired from initial poses of robots and vice versa, considering initial pose is represented as origin in the map. This holds usually true for existing \gls{SLAM} algorithms.

Problem arises when initial poses of robots are not known to merging algorithm. When robots are starting from the same place, initial relative positions can be measured with basic equipment either present on the site or mounted on robots. When robots are starting from distant locations measuring initial poses might require more sophisticated equipment, as widely available hardware such as \gls{GPS} sensors might not provide required precision. In indoor environments determining starting positions might be even more challenging.

Because of these difficulties, algorithms that can estimate transformations between maps and therefore merge maps without knowledge of initial positions have been developed. A comprehensive survey of map-merging techniques is done in~\cite{Lee2012}, which classifies algorithms as Direct Map Merging and Indirect Map Merging.

\section{Direct map merging}

Direct map merging algorithm relies on sensors to directly compute transformation between reference frame of robots. This includes techniques relying on direct robot rendezvous~\cite{Zhou2006} and similar. This techniques can provide a highly accurate transformation estimate, but are limited by relying on specific conditions to estimate this transformation such as the robot rendezvous or encountering specific landmark. These systems also exchange specific data related to selected feature to estimate transformation or even require control over robot~\cite{Konolige2003}. Implemented solutions therefore tend to be monolithic, because they rely on exchange of custom messages and specific sensors data, which makes them hard to implement especially for heterogeneous robots.

\section{Indirect map merging}

Indirect Map Merging algorithm uses overlapping areas in maps to estimate transformation between maps. Merging maps based on map data only naturally creates a simple interface between robots and works well for heterogeneous groups of robots. Each robot only need to expose map in common format, sensor equipment mounted on robots may be different as well as \gls{SLAM} algorithm used for creating map. This approach is usually more flexible than direct map merging, robots may visit overlapping area at different time (in contrast to robot rendezvous) and this approach is also absolutely passive requiring no control over robot.

Wide range of techniques has been employed for indirect map merging. \cite{Lee2012} mentioned in this category techniques based on scan matching algorithms. These algorithms are working with \gls{SLAM} representation of maps and they are usually tightly coupled with specific \gls{SLAM}. This makes them share some disadvantages (such as difficult scalability through heterogeneous systems) with direct map merging algorithms. Algorithms in this category include \cite{Wang2012} work based on visual \gls{SIFT} features and topology nodes, approach of~\cite{Tungandi2010} combining omnidirectional vision and laser scans, \cite{Cunningham2012} using graph \gls{SLAM} with condensed maps and non-linear constrained optimization to acquire transformation between maps. Data \gls{SLAM} algorithms uses to represent map may be too big to exchange all of them between robots, this was dressed by~\cite{Lazaro2013}, which uses condensed measurements and multi robot graph \gls{SLAM}.

Only a few algorithms works exclusively with portable map representation (maps represented as two-dimensional occupancy grids), despite this promises better scalability and by design supports heterogeneous multi-robot systems. Works using occupancy grids include spectra-based approach of~\cite{Carpin2008}, \cite{Lee2011} combines his approach with concept of virtual supporting lines to merge custom sparse maps of infrared features. \cite{martins2013mrslam} used algorithm of~\cite{MapstitchROS} based on image features to merge maps of $2$ robots, which is limitation of this algorithm.

The novel algorithm for map merging presented in section~\ref{chap:mergingalgorithm} is using only occupancy grids to produce the merged map. This algorithm is inspired by image stitching algorithms for creating photo panoramas. It is designed to merge maps from arbitrary number of robots, overcoming the limitation of both~\cite{MapstitchROS} and \cite{Carpin2008}. It employs \gls{RANSAC} for robust transform estimation and uses probability model to evaluate confidence of estimated transformation. Matching phase is accelerated using parallel hierarchical clustering trees proposed by~\cite{Muja2012} so that algorithm scales well to large number of robots.

% http://journals.cambridge.org/article_S026357471100049X
