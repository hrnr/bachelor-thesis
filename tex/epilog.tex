\chapwithtoc{Conclusion}

Presented map-merging algorithm can efficiently work with artificial number of robots, scales to large multi-robot systems and is designed with parallel processing in mind. Algorithm is suitable for heterogeneous multi-robot teams and is easily deployable with various \gls{SLAM} algorithms.

The algorithm is implemented in \texttt{multirobot\_map\_merge} \gls{ROS} package. This implementation is flexible, poses low requirements on participating robots, does not depend on any particular communication between robots and does not have any presumptions on underlying mechanisms of \gls{SLAM} algorithms used by robots. Those properties allows easy deployment in \gls{ROS} environment for both existing systems and systems build from scratch. Performance of the implemented initial pose estimation algorithm is sufficient to merge map from large number of robots (more than $30$) on a laptop-grade processor. For large number of robots map compositing is the bottleneck for map-merging node, this will be addressed in the next version of the map-merging node.

The map-merging algorithm has been evaluated in the simulation and showed reliable estimates for maps with enough overlapping area. Necessary overlaps range from $5$ to $7$ rooms in a feature-poor simulation. Behaviour for online merging has been studied and implementation has been adapted to produce a consistent map when there is not enough overlapping space.

Evaluation with physical robots is yet to be done. I expect performance for real-world applications to be comparable to the experiments in the simulator or better, because physical environments are usually more feature-rich than scenarios in the simulator. This thesis also focused on map-merging for two-dimensional maps, map-merging for tree-dimensional maps remains an open problem in robotics.
