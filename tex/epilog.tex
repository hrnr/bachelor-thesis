\chapwithtoc{Conclusion}

Map-merging algorithm presented in Section~\ref{chap:mergingalgorithm} can efficiently work with arbitrary number of robots. It scales well to large multi-robot systems and is designed with parallel processing in mind. The algorithm is suitable for heterogeneous multi-robot swarms and is easily deployable with various \gls{SLAM} algorithms.

Proposed algorithm is based on image processing techniques and uses \gls{OpenCV} through implementation. I have proposed a project implementing presented affine transformation estimation for \gls{OpenCV} stitching pipe. This project have been accepted for Google Summer of Code 2016.

The algorithm is implemented in \texttt{multirobot\_map\_merge} \gls{ROS} package presented in Section~\ref{sec:map_merge-package}. This implementation is flexible, imposes low requirements on participating robots, does not depend on any particular communication between robots and does not have any presumptions on underlying mechanisms of \gls{SLAM} algorithms used by robots. Those properties allows easy deployment in \gls{ROS} environment for both existing systems and systems build from scratch. Performance of the implemented initial pose estimation algorithm is sufficient to merge maps from large number of robots (more than $30$) on a laptop-grade processor. For large number of robots map composing is the bottleneck for map-merging node, this will be addressed in the next version of the map-merging node.

For purposes of evaluation I have created \texttt{explore\_lite} package presented in Section~\ref{sec:explore_lite-package} for autonomous exploring. Both packages have been accepted by the \gls{ROS} project and are included in the \gls{ROS} binary distribution.

During development I have make contributions to the \gls{ROS} project including changes to the \gls{ROS} navigation stack and \gls{ROS} wiki. Changes for the \gls{ROS} navigation stack have been accepted and will be included in the upcoming \gls{ROS} release, changes to the \gls{ROS} wiki (support for HTML5 videos) are already online.

The map-merging algorithm has been evaluated in the simulation and showed reliable estimates for maps with enough overlapping area. Necessary overlaps range from $5$ to $7$ rooms in a feature-poor simulation. Behaviour for online merging has been studied and implementation has been adapted to produce a consistent map when there is not enough overlapping space.
