\chapwithtoc{Conclusion}

Presented map-merging algorithm can efficiently work with artificial number of robots, scales to large number robots and is designed with parallel processing in mind. Algorithm is suitable for heterogeneous multi-robot systems and is easily deployable with various \gls{SLAM} algorithms.

Algorithm is implemented in \texttt{multirobot\_map\_merge} \gls{ROS} package. Implementation is flexible, poses low-requirements on participating robots, does not depend on any particular communication between robots and does not have any presumptions on underlying mechanisms of \gls{SLAM} algorithms used by robots. Those properties allows easy deployment in \gls{ROS} environment for both existing systems and systems build from scratch. Performance of the implemented map-merging algorithm (initial pose estimation) is sufficient to merge map from large number of robots (more than $30$) on laptop-grade processor. For large number of robots map compositing is the bottleneck for map-merging node, this will be addressed by using vectorized code for compositing.

Algorithm was evaluated in simulation and showed reliable estimates for maps with enough overlapping area. Necessary overlaps range from $5$ to $7$ rooms in feature-poor simulation. Behaviour for online merging has been studied and implementation was adapted to produce small but consistent map when there is not enough overlapping space.

Evaluation with physical robots is yet to be done. I expect performance for real-world applications should be comparable to experiments in simulator or better, because physical environments are usually more feature-rich than scenarios in simulator. This thesis also focused on map-merging for two-dimensional maps, map-merging for tree-dimensional maps remains an open problem in robotics.
