\chapwithtoc{Introduction}

Multi-robot exploring teams have several advantages over single robot. When properly coordinated performace the such system is higher, single point of failure is eliminated in fully distributed systems and multiple robots can possibly do tasks single robot could not. Multi-robot systems are not limited to exploration tasks applications include payloads delivery, military tasks, search and rescue in disaster areas and patroling. Multi-robot teams can overcome imperfections in underlying navigation and mapping algorithms especially when using heterogeneous robot teams, where stucked robot can be replaced by robot which uses different algorithm.

In multi-robot systems shared map is required for effective coordination. Map-merging algrotihm producing global map is therefore essential technique for multi-robot systems. Map-merging is a challenging problem, especially in heterogeneous systems.

Key aspects of map-merging and related works are discussed in section~\ref{chap:map-merging-intro}. In section~\ref{chap:mergingalgorithm} I present a novel map-merging algorithm based on image stitching techniques, which can work with heterogeneous multi-robot teams and is scalable to large number of robots.

Section~\ref{chap:ros-packages} presents ready-to-use \gls{ROS} packages implementing presented map-merging algorithm and frontier-based autonomous exploration. Section~\ref{chap:evaluation} discusses performace of presented map-merging algorithm achived in several simulation experiments.

Documentation for \gls{ROS} packages presented in section~\ref{chap:ros-packages} is attached as appendices~\ref{chap:map_merge-doc}, \ref{chap:explore-doc}.
