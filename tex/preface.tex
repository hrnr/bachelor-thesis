\chapwithtoc{Introduction}

Multi-robot exploring swarms have several advantages over a single robot. When properly coordinated performace of the multi-robot system is higher and multiple robots can possibly do tasks single robot could not. In fully distributed systems single point of failure is eliminated.

Multi-robot swarms can overcome imperfections in underlying navigation and mapping algorithms especially when using heterogeneous robot swarms, where one stucked robot can be replaced by another robot which uses different algorithm.

This work focuses on map-merging, which is a challenging problem, especially in heterogeneous robot swarms. In multi-robot systems shared map is required for effective coordination. Map-merging algorithm producing global map is therefore essential component of multi-robot systems.

This text is structured as follows: Key aspects of map-merging and related works are discussed in Section~\ref{chap:map-merging-intro}. In Section~\ref{chap:mergingalgorithm}, I present a novel map-merging algorithm based on image stitching techniques, which can work with heterogeneous multi-robot swarms and is scalable to large number of robots.

Section~\ref{chap:ros-packages} presents ready-to-use \gls{ROS} packages implementing presented map-merging algorithm and frontier-based autonomous exploration. Section~\ref{chap:evaluation} discusses performace of presented map-merging algorithm achieved in several simulation experiments.

Documentation for \gls{ROS} packages presented in Section~\ref{chap:ros-packages} is attached as Appendices~\ref{chap:map_merge-doc}, \ref{chap:explore-doc}.
