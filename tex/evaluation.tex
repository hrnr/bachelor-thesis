\chapter{Evaluation}
\label{chap:evaluation}

To evaluate presented map-merging algorithm~\ref{chap:mergingalgorithm} and test performance of implemented map-merging node~\ref{sec:map_merge-package} I used $2$ data sources. First data source was simulation running several P3DX robots. Map-merging node was running through the whole exploring session testing online behaviour of the algorithm. Second data source were maps produced by \gls{SLAM} on \gls{MIT} dataset \#reference here\#. \gls{MIT} dataset produced using a single PR2 robot starting from different locations. Although this data does not come from multi-robot mapping, it is possible to test offline merging performance.

\section{Simulation setup}

I used \gls{VREP} simulator for experiment. All simulated robots were Pioneers P3-DX, which formed a homogeneous exploring team. Robots were setup using \texttt{p3dx\_robot} \# reference here \# package, which also setups \gls{SLAM} and navigation for robots. Robots were using \texttt{hector\_slam} package providing \gls{SLAM} algorithm and \texttt{move\_base} package providing navigation for robots.

Cluster of 5 computers was formed to run simulator, robots, map-merging and exploring nodes. \gls{ROS} network was configured across all workstations using a single \gls{ROS} master running \texttt{roscore}. Every robot was using its own \gls{ROS} namespace for topics and was using a prefix for published \texttt{tf} frames to allow running multiple robots under the same \gls{ROS} master. This setup is well supported in \texttt{p3dx\_robot} package.

While \gls{VREP} is powerful and feature-rich simulator, its usage for multi-robot simulation have some limitations. First of all, \gls{VREP} support for headless mode (running without graphical environment) is not complete. Virtual framebuffer or similar technology is required, which adds performance overhead. Further \gls{VREP} does not scale properly to large number of threads, limiting number of robots for which simulation runs at bearable speed. For this reason it wasn't possible  to test more than 4 robots with this setup.

\section{\gls{MIT} dataset}

\gls{MIT} dataset are data available as rosbags from \#reference here\#. Data comes from mapping multi-floor \gls{MIT} building with PR2 robot. I have used only datasets from the second floor.

For all rosbags I have created maps using the \texttt{hector\_slam} package. It is same \gls{SLAM} algorithm, which was used in simulation. This resulted in $36$ occupancy grid maps with $2048 \times 2048$ cells each.

Produced maps has been statically served in \gls{ROS} with running map-merging node. This setup is therefore limited to test offline merging, but allows a greater number of maps to participate in merging.

It is important to node that presented maps were created by a single robot in multi-session mapping. It is not a result of multi-robot mapping, although the robot initial positions vary between sessions and produced maps are similar to maps we would expect from multi-robot mapping.

\section{Minimal overlapping area}

